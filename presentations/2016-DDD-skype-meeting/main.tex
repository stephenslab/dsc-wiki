\input{preamble.uchicago}
\usepackage{tikz}
\usepackage{multirow}
\usepackage{algorithm2e}
\usepackage{dsfont}
\usepackage{mathtools}
\usepackage{epstopdf}
\usepackage{xspace}
\usepackage[export]{adjustbox}
\SetKwRepeat{Do}{do}{while}%
\newcommand{\etal}{{\em et al.\@\xspace}}
\setbeamertemplate{sections/subsections in toc}[default]
\setbeamertemplate{caption}{\raggedright\insertcaption\par}
\renewcommand{\UrlFont}{\color{ucMaroon}\footnotesize}

%% Colored Blocks
\newcommand{\CBc}[2]{\tikz \node[circle,scale=0.75,color=white,fill=#1]{\textbf{#2}};}
\newcommand{\CBr}[2]{\tikz \node[rectangle,scale=0.9,color=white,fill=#1]{\textbf{#2}};}

%% https://github.com/aschn/gnuplot-colorbrewer/blob/master/qualitative/Set1.plt
\definecolor{s1red}{HTML}{E41A1C}
\definecolor{s1blue}{HTML}{377EB8}
\definecolor{s1green}{HTML}{4DAF4A}
\definecolor{s1purple}{HTML}{984EA3}
\definecolor{s1orange}{HTML}{FF7F00}
\definecolor{s1yellow}{HTML}{FFFF33}
\definecolor{s1brown}{HTML}{A65628}
\definecolor{s1pink}{HTML}{F781BF}
%% https://github.com/aschn/gnuplot-colorbrewer/blob/master/qualitative/Dark2.plt
\definecolor{d2teal}{HTML}{1B9E77}
\definecolor{d2orange}{HTML}{D95F02}
\definecolor{d2lilac}{HTML}{7570B3}
\definecolor{d2magenta}{HTML}{E7298A}
\definecolor{d2green}{HTML}{66A61E}
\definecolor{d2banana}{HTML}{E6AB02}
\definecolor{d2tan}{HTML}{A6761D}
\definecolor{d2gray}{HTML}{666666}

\hypersetup{colorlinks,urlcolor=s1blue} % href's are correct, but navigation links are magenta
%% Sequence data sources
\newcommand{\dscr}{\CBr{d2orange}{DSCR}}
\newcommand{\dscii}{\CBr{d2green}{DSC2}} 
\makeatletter
    \newenvironment{withoutheadline}{
        \setbeamertemplate{headline}[default]
        \def\beamer@entrycode{\vspace*{-\headheight}}
    }{}
\makeatother

\title{Dynamic Statistical Comparisons}
\titlegraphic{\includegraphics[width=3cm]{figs/uchicago}}
\author[GW]{Gao Wang \& Matthew Stephens}
\institute[UC]{
 Stephens Lab \\ 
 University of Chicago
}
\date{September 28, 2016}

\begin{document}

%% Title
\begin{frame}[plain]
  \titlepage
  %% \footnotesize{Copyright (c) 2015 by Raman A.~Shah.\\
  %% \href{https://creativecommons.org/licenses/by-nc-sa/3.0/legalcode}
  %%      {Creative Commons BY-NC-SA 3.0 Unported}.\\
  %%  \href{https://github.com/ramanshah/intermediate\_git}
  %%       {https://github.com/ramanshah/intermediate\_git}}
\end{frame}

%% Outline
%%\begin{frame}[plain]
%%\tableofcontents
%%\end{frame}

%%
\section{Introduction}
%%
\begin{frame}{DSC aids in reproducible research}
  Dynamic Statistical Comparisons (DSCs):
\begin{itemize}
\item Comparisons typically performed is suboptimal in many ways
\item The idea of DSC is an attempt to make statistical comparisons
  \textbf{easily extensible} and \textbf{reproducible}
\end{itemize}
\bigskip
Our desired products: 
\begin{itemize}
\pause \item A \textit{platform} to make it \textbf{simple}, even \textbf{fun} to carry out DSC
\item A \textit{DSC repository} to facilitate research in Stephens Lab
\begin{itemize}
  \item ... and the entire research community
\end{itemize}
\end{itemize}
\end{frame}

\begin{frame}{DSC2: the new DSC platform}
  \dscr
\begin{itemize}
  \item Our first attempt to a DSC platform in the \texttt{R} language
  \item A successful proof-of-concept implementation to DSC model
  \item Lacks flexibility and capacity for complex and large scale DSC 
\end{itemize}
\bigskip
  \dscii
\begin{itemize}
  \item Our recent attempt to a multi-language DSC platform 
    \begin{itemize}
      \item Mix-and-match \texttt{R}, \texttt{Python} and \texttt{Shell} programs 
      \item Assembly of statistical procedures like LEGO 
      \item Engineered by modern workflow management system standards
     \end{itemize}
\end{itemize}
\end{frame}
\begin{frame}{DSC2: key idea illustrated}
  \centering \includegraphics[width=\textwidth]{figs/dsc1}
\end{frame}
\begin{withoutheadline}
\begin{frame}
  \centering \includegraphics[height=\textheight]{figs/dsc2}
\end{frame}
\end{withoutheadline}
\begin{withoutheadline}
\begin{frame}
  \centering \includegraphics[width=\textwidth]{figs/dsc3}
\end{frame}
\end{withoutheadline}
\begin{withoutheadline}
\begin{frame}
  \centering \includegraphics[width=\textwidth]{figs/dsc4}
\end{frame}
\end{withoutheadline}
\begin{withoutheadline}
\begin{frame}
  \centering \includegraphics[width=\textwidth]{figs/dsc5}
\end{frame}
\end{withoutheadline}
\begin{frame}{User interface \& DSC browser}
\begin{itemize}
  \item A command line tool that generates HTML reports
\end{itemize}
  \centering \includegraphics[width=0.9\textwidth]{figs/dsc6}
\begin{itemize}
  \item DSC example from \href{http://www.bioinformatics.org/labnotes/dr-tree/dsc/20160630/benchmark.html}{a lab project}
  \item ... and the
    \href{http://www.bioinformatics.org/labnotes/dr-tree/figures/20160630/simulation-brownian-20160630.html}{benchmark generated}
\end{itemize}
\end{frame}
\begin{frame}{Next steps}
  \textbf{Improved DSC report and visualization}
  \begin{itemize}
    \item Make it even more fun to build DSC
  \end{itemize}
  \textbf{Support for cluster computing}
  \begin{itemize}
    \item Implement Directed Acyclic Graph (DAG) for DSC jobs
    \item Job management and signature tracking via \texttt{redis}
  \end{itemize}
  \textbf{Application to Stephens Lab projects}
  \begin{itemize}
    \item Adapt existing projects to DSC2
    \item Advocate collaborations via DSC2 within the lab
  \end{itemize}
\end{frame}
\end{document}